% Copyright 2011 David W. Hogg (NYU).
% All rights reserved.

\documentclass[12pt]{article}

\newcommand{\dd}{\mathrm{d}}

\begin{document}\sloppy\sloppypar\raggedbottom

\textsl{This document is a draft dated 2011-11-09.  It incorporates
  direct contributions from Joseph~W.~Richards (Berkeley),
  David~W.~Hogg (NYU), James~P.~Long (Berkeley), and
  Daniel~Foreman-Mackey (NYU) and indirect contributions from many
  others.}

\vspace{1ex}

A photometric survey (think of it as operating in a single photometric
bandpass for now) scans over the celestial position of a particular
variable star at a large number of times $t_i$.  At each of these
times, the imaging data are analyzed by (not awesome) software, which
treats each scan as a \emph{completely independent} survey.  That is,
when analyzing the data from time $t_i$, none of the information from
any of the other times is used in any way.

At each time $t_i$, the star is either detected ($q_i=1$) by the
software or not ($q_i=0$).  When it is detected, the software returns
a (possibly bad) flux value $f_i$ and (likely bad) uncertainty
variance $s_i^2$.  When the source is \emph{not} detected ($q_i=0$),
nothing is reported---in this case, because the software is not
awesome, it doesn't see any reason to report anything at all, since on
the non-detect passes, it has no inkling that this patch of the sky is
interesting in any way.  Welcome to the desert of the real.

Beyond this, there are two additional issues.  The first is that each
night of imaging is different in an unreported and unknown way.  That
is, there is different transparency, sky brightness, and point-spread
function.  So the detection limit, or completeness level, or censoring
of the catalog is different every night.  The second issue is that
either because we are getting the data from a non-generous source, or
because the conditions change rapidly and unpredictably, we can't
analyze all the sources in a finite patch simultaneously.  For each
variable star of interest, we \emph{only} get the data on that star
itself.

To recap, all we get at each epoch $t_i$ is a bit $q_i$ and, when
$q_i=1$, two data $f_i$ and $s_i^2$.  We don't get any information
about the upper limits or detection thresholds at the times $t_i$ when
$q_i=0$, and we don't know anything about the data quality (catalog
veracity) at any epoch.  For notational convenience we assemble all
the data into an enormous set $D$ given by
\begin{eqnarray}\displaystyle
D &\equiv& \{D_i\}
\\
D_i &\equiv& (q_i, f_i, s_i^2)
\quad ,
\end{eqnarray}
where the $f_i$ and $s_i^2$ data will be randomly generated or
meaningless at times at which the $q_i=0$.

We don't have access to all the information we need to use responsibly
the detections and non-detections in model fitting and inference.  We
have to instantiate latent variables to stand in for the missing
information.  For example, since we don't know the censoring threshold
or completeness limit for any observation $i$, we have to make it a
model parameter $b_i$.  Because we don't believe the reported
uncertainty variances $s_i^2$, we have to add parameters that model
the departure of the true uncertainty variance $\sigma_i^2$ from the
reported variance, and so on.

For the limited purposes here, the variable star will be periodic with
angular frequency $\omega$.  The expected brightness $\mu_i$ of the
star at time $t_i$ will be given by a linear combination of periodic
functions something like
\begin{eqnarray}\displaystyle
\mu_i &=& \sum_k A_k\,x_k(t_i|\omega)
\quad ,
\end{eqnarray}
where the $A_k$ are coefficients and the $x_k$ are functions.  This
can be generalized easily.  In addition to this expected brightness,
we assume that there is some either model uncertainty or stochastic
variation, leading to a \emph{model variance} $s_\mu^2$ at each point
(assumed constant but easily generalized).

If a source is detected at epoch $t_i$, we assume that its observed
flux $f_i$ is related to the expected flux $\mu_i$ by a Gaussian with
variance that has measurement and model contributions
\begin{eqnarray}\displaystyle
p(f_i|q_i,\sigma_i,\theta,I) &=& \left\{\begin{array}{ll}
  1 & \mbox{for $q_i=0$} \\
  N(f_i|\mu_i,s_\mu^2+\sigma_i^2) & \mbox{for $q_i=1$}
\end{array}\right.\label{eq:poff}
\\
\theta &\equiv& (\omega, \{A_k\}, s_\mu^2, \cdots)
\\
I &\equiv& (\{t_i\}, \mbox{assumptions})
\quad ,
\end{eqnarray}
where we have made the prior information about observation times $t_i$
and other assumptions explicit in the prior information blob $I$, and
we have a parameter blob $\theta$, which we will specify completely
below.  Note that the flux datum is ignored when the source is
undetected.

Similarly for the reported uncertainty variance $s_i^2$.  We don't
assume that it is correct in any case, but when the source is
detected, we assume that it is connected to the true uncertainty
$\sigma_i$ by a probabilistic likelihood, also Gaussian
\begin{eqnarray}\displaystyle
p(s_i|q_i,\sigma_i,\theta,I) &=& \left\{\begin{array}{ll}
  1 & \mbox{for $q_i=0$} \\
  N(s_i|\sigma_i,v_\sigma) & \mbox{for $q_i=1$}
\end{array}\right.
\\
\theta &\equiv& (\omega, \{A_k\}, s_\mu^2, v_\sigma, \cdots)
\quad ,
\end{eqnarray}
where we have added a model parameter $v_\sigma$, which represents the
variance in the distribution of reported uncertainties given the true
uncertainty; implicitly $v_\sigma$ is a component of $\theta$.  Again,
we are ignoring the variance datum when the source is undetected.

Under the assumption that the inclusion (or not) of the source in the
catalog is based on (or very strongly related to) the measured flux
(rather than the true flux), the natural likelihood for the bit $q_i$,
on which the above likelihoods are conditioned is
\begin{eqnarray}\displaystyle
P(q_i|b_i,\sigma_i,\theta,I) &=& \left\{\begin{array}{ll}
  \displaystyle\int_0^{b_i} N(f|\mu_i,s_\mu^2+\sigma_i^2)\,\dd f & \mbox{for $q_i=0$} \\
  \displaystyle\int_{b_i}^\infty N(f|\mu_i,s_\mu^2+\sigma_i^2)\,\dd f & \mbox{for $q_i=1$}
\end{array}\right.\label{eq:pofq}
\quad,
\end{eqnarray}
where the integrals end at an observed-flux limit $b_i$, which (as
mentioned above) is also a model parameter.  The motivation for this
expression is that it imagines all possible observations $f$ of the
star at time $t_i$ given the model and uncertainties, and finds the
fraction that would make the (imagined) detection cut.

We are going to product together these likelihoods (for $q_i$, $f_i$,
and $s_i^2$); this looks like it is using the flux twice, once in the
PDF for $f_i$ in expression (\ref{eq:poff}) and once in the
probability for $q_i$ in expression (\ref{eq:pofq}).  It is not,
however, because in the probability for $q_i$ the flux $f$ that
appears is an integration variable, not an observation or datum.  We
treat the detections or non-detections as being ``prior'' to the
measurement.  This is the weakest plank in our otherwise impenetrable
fortress of solitude.

We are going to treat the true detection limit $b_i$ for each
observation, and its true uncertainty $\sigma_i$---two parameters
\emph{per data point}---as nuisance parameters.  We want to
marginalize them out.  This requires priors on these.  Again we choose
Gaussians for simplicity, but let the means and variances of these
Gaussians be new model parameters---hyperparameters if you wish.
\begin{eqnarray}\displaystyle
p(D_i|\theta,I) &=&
  \int P(D_i|b_i,\sigma_i,\theta,I)\,p(b_i|\theta)\,p(\sigma_i|\theta)\,\dd b_i\,\dd\sigma_i
\\
p(D_i|b_i,\sigma_i,\theta,I) &=&
  P(q_i|b_i,\sigma_i,\theta,I)\,p(f_i|q_i,\sigma_i,\theta,I)\,p(s_i|q_i,\sigma_i,\theta,I)
\\
p(b_i|\theta) &=& N(b_i|B,V_B)
\\
p(\sigma_i|\theta) &=& N(\sigma_i|S,V_S)
\\
\theta &\equiv& (\omega, \{A_k\}, s_\mu^2, v_\sigma, B, V_B, S, V_S)
\quad,
\end{eqnarray}
where we have (at last) explicitly assembled all the variable-star
parameters and hyperparameters into the big parameter vector $\theta$.
The huge sets of nuisance parameters $\{b_i\}$ and
$\{\sigma_i\}$ don't remain, because by integration we
removed them.

Finally and it goes without saying that if we treat the data points as
independent (as we are free to do, given the non-awesomeness of the
software data source), we have that
\begin{eqnarray}\displaystyle
p(D|\theta,I) &=& \prod_i p(D_i|\theta,I)
\quad.
\end{eqnarray}
This is the likelihood for variable-star parameters given the entire
data set (all the measurements and non-detections of this star from
all the epochs, as delivered by the untrustworthy robots).  It
``correctly'' or at least ``justifiably'' uses all of the information
available, without making strong assumptions about the survey or its
veracity.

\end{document}
