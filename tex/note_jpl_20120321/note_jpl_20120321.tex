\documentclass[10pt]{article}
\usepackage{verbatim, amsmath,amssymb,amsthm,graphicx}
\usepackage[margin=.5in,nohead,nofoot]{geometry}
\usepackage{sectsty}
\usepackage{float}
\sectionfont{\normalsize}
\subsectionfont{\small}

\title{}
\date{}
\author{}
\newtheorem{theorem}{Theorem}[section]
\newtheorem{definition}{Definition}[section]
\newtheorem{example}{Example}[section]

\newcommand{\argmin}[1]{\underset{#1}{\operatorname{argmin}}\text{ }}
\newcommand{\argmax}[1]{\underset{#1}{\operatorname{argmax}}}
\newcommand{\minimax}[2]{\argmin{#1}\underset{#2}{\operatorname{max}}}
\newcommand{\bb}{\textbf{b}}

\newcommand{\Var}{\text{Var }}
\newcommand{\Cov}{\text{Cov }}


\newenvironment{my_enumerate}{
  \begin{enumerate}
    \setlength{\itemsep}{1pt}
    \setlength{\parskip}{0pt}
    \setlength{\parsep}{0pt}}{\end{enumerate}
}



% Alter some LaTeX defaults for better treatment of figures:
% See p.105 of [yas] elisp error!TeX Unbound'' for suggested values.
% See pp. 199-200 of Lamport's [yas] elisp error!LaTeX'' book for details.
%   General parameters, for ALL pages:
\renewcommand{\topfraction}{0.9}% max fraction of floats at top
\renewcommand{\bottomfraction}{0.8}% max fraction of floats at bottom
%   Parameters for TEXT pages (not float pages):
\setcounter{topnumber}{2}
\setcounter{bottomnumber}{2}
\setcounter{totalnumber}{4}     % 2 may work better
\setcounter{dbltopnumber}{2}    % for 2-column pages
\renewcommand{\dbltopfraction}{0.9}% fit big float above 2-col. text
\renewcommand{\textfraction}{0.07}% allow minimal text w. figs
%   Parameters for FLOAT pages (not text pages):
\renewcommand{\floatpagefraction}{0.7}% require fuller float pages
% N.B.: floatpagefraction MUST be less than topfraction !!
\renewcommand{\dblfloatpagefraction}{0.7}% require fuller float pages

% remember to use [htp] or [htpb] for placement


\begin{document}
\section*{Initial Test of Nat's Code on Miras and RR Lyrae - JPL - March 21}

I ran Nat's censored code on all our Miras and RR Lyrae both with censored observations and without censored observations.

\subsection{Miras}
\begin{itemize}
\item of the 1720 Miras, 247 had different period estimates (by more than 1\%) after including the censored observations
\item from visual inspectation almost all these 247 had the correct period after using the censored data
\item for the 247, using censored data nearly always doubled the period estimate (see plots below)
\end{itemize}



\noindent For RR Lyrae:
\begin{itemize}
\item i like RR lyraes
\end{itemize}

- density of miras periods w / wo censored obs, density of rr lyrae periods w / w/o censored 
- trouble getting unnormalized amplitude out of code
- may be issue with selection of miras, other interesting patterns may arise if we run code on non-miras




\begin{figure}[h]
  \begin{center}
    \begin{includegraphics}[height=4in,width=4in]{50.pdf}
      \caption{Mira where without using censored data the period is half of the estimate using censored. \textbf{period w/o censored = half period with censored} is the most common scenario among the Miras.}
    \end{includegraphics}
  \end{center}
\end{figure}


\begin{figure}[h]
  \begin{center}
    \begin{includegraphics}[height=4in,width=4in]{52.pdf}
      \caption{Another Mira.}
    \end{includegraphics}
  \end{center}
\end{figure}



\begin{figure}[h]
  \begin{center}
    \begin{includegraphics}[height=4in,width=4in]{96.pdf}
      \caption{And a third Mira.}
    \end{includegraphics}
  \end{center}
\end{figure}



\end{document}
